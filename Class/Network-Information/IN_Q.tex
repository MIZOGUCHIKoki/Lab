\RequirePackage{luatex85}
\documentclass[paper=a4,10pt,twocolumn]{jlreq}
\usepackage{amsmath,amsfonts,amssymb,mathtools,ascmac,bm,fancybox,calc,array}
\usepackage[top=20truemm,bottom=20truemm,left=15truemm,right=15truemm]{geometry}
\usepackage{graphicx,color}
\usepackage{tikz,listings,wrapfig,float,xcolor}
\usepackage{url,subcaption,multirow,framed}
\usepackage[unicode,hidelinks,pdfusetitle]{hyperref}
\usepackage{luatexja-fontspec,lltjext}
\usepackage{tcolorbox,comment}
\hypersetup{
    colorlinks=true,
    citecolor=black,
    linkcolor=black,
    urlcolor=blue
    }
    
    \usetikzlibrary{intersections,calc,arrows.meta,backgrounds,shapes.geometric,shapes.misc,positioning,fit,graphs,arrows}
    \tcbuselibrary{raster,skins,breakable,theorems}
    \setlength{\columnsep}{5mm}
    \columnseprule=0.1mm
    \ltjsetparameter{jacharrange={-2}} %日本語以外を欧文扱い

    \renewcommand{\indent}{\hspace{1\zw}}
    \setlength{\parindent}{1\zw}
    
    \renewcommand{\thefootnote}{*\arabic{footnote}}
    \renewcommand{\figurename}{Fig\ }
    \renewcommand{\tablename}{Tbl}
    \newcommand{\figref}[1]{Fig\ \ref{#1}}
    \newcommand{\tabref}[1]{Tbl\ \ref{#1}}

\makeatletter
    \renewcommand{\thefigure}{%
    \thesection.\arabic{figure}}
    \@addtoreset{figure}{section}

    \renewcommand{\thetable}{%
    \thesection.\arabic{table}}
    \@addtoreset{table}{section}

    \@addtoreset{lstlisting}{section}
\makeatother


\title{\vspace{-2cm}情報ネットワーク基礎}
\author{}
\date{}
\begin{document}
\twocolumn[
    \begin{center}
        {\LARGE \vspace{-1cm}情報ネットワーク基礎}
    \end{center}
    \begin{flushright}
        {\Large K.MIZOGUCHI}
    \end{flushright}
    \dotfill\\

]
\newcommand{\ci}{◯}
\newcommand{\bt}{×}
\section*{クイズ1}
\subsection*{問1}
回線交換の特徴として正しいものはどれか.
\begin{description}
    \item[\bt] 混雑時には遅延が発生する.
    \item[\ci] 開始後は情報を送っていない時間も回線を専有する.
    \item[\ci] 通信中は回線を専有する.
    \item[\ci] 通信開始時に経路が決まる.
\end{description}
\subsection*{問2}
パケット交換の特徴として正しいものはどれか.
\begin{description}
    \item[\bt] パケットの到着順序が保証されている.
    \item[\ci] 混雑時に遅延が発生する.
    \item[\ci] 回線を専有しない.
    \item[\ci] 情報をパケットに分割して送る.
    \item[\ci] 中継地点ではその時の状況で,適切な方へ送る.
\end{description}
\subsection*{テストに出る!!}
\begin{itemize}
    \item \textbf{7. アプリケーション層}\\
          アプリケーション毎のサービス提供機能
    \item \textbf{6. プレゼンテーション層}\\
          データの表現形式
    \item \textbf{5. セッション層}\\
          アプリケーション同士の会話制御
    \item \textbf{4. トランスポート層}\\
          両端コンピュータでの処理,輻輳抑制,信頼性提供
    \item \textbf{3. ネットワーク層}\\
          途中の中継処理,経路制御
    \item \textbf{2. データリンク層}\\
          直接接続された機器間のデータフレーム転送.
    \item \textbf{1. 物理層}\\
          データを物理的信号に変換,ケーブルやコネクタの形状.
\end{itemize}
\newpage
\section*{クイズ2}
\subsection*{問1}
有線イーサネットで用いられている CSMA/CD の手順を説明している.
\begin{description}
    \item[\ci] 媒体が空いていることを確認できたら送信を開始する.
    \item[\ci] 送信しながら,他からの信号を確認する.
    \item[\bt] \textbf{衝突(Collision)を検出したら,送信を中断して一定時間だけ待つ.}
    \item[\ci] \textbf{待った後に最初から再送する}.
    \item[\ci] 再衝突が起きたら待ち時間を倍にして繰り返す.
\end{description}
\subsection*{問2}
次のイーサネットの規格で,ケーブルにツイストペアケーブルを用いるのはどれか.
\begin{description}
    \item[\ci] 100BASE-TX("T"が入っているもの.)
    \item[\bt] 100BASE-FX
    \item[\bt] 1000BASE-SX
    \item[\bt] 1000BASE-LX
    \item[\bt] 10GBASE-SR
\end{description}
マルチとシングルでは,\textbf{シングル}のことが優れている.(細いから)
\subsection*{問3}
次のうち,無線LANの規格名となっているのはどれか.
\begin{description}
    \item[\ci] IEEE 802.11ax
    \item[\bt] IEEE 802.11cx
    \item[\bt] IEEE 802.3ac
    \item[\bt] IEEE 802.5n
    \item[\bt] IEEE 802.3be
\end{description}
無線LANの規格は,\verb|IEEE 802.11XX|のもの.種類は以下.
\begin{center}
    \verb|b, a, g, n, ac, ax, be|
\end{center}
\newpage
\section*{クイズ3}
\subsection*{問1}
IPアドレスに関する記述で正しのはどれか.
\begin{description}
    \item[\ci] IPv4のアドレスは32Bit.
    \item[\bt] IPv6のアドレスは64Bit.(正しくは128Bit)
    \item[\ci] IPv4ではネットワークアドレス部を任意のビット数に設定できる.
    \item[\ci] ブロードキャストアドレスは,IPアドレスのホストアドレス部を全て\verb|1|にしたアドレスである.
    \item[\ci] ネットワークアドレスは,IPアドレスのホストアドレス部を全て\verb|0|にしたアドレスである.
\end{description}
\subsection*{問2}
IPv4で,IPアドレスが\verb|222.229.64.8|のホストが接続されているネットワークのブロードキャストアドレスはいくつか?
ただし,サブネットマスクは20Bitとする.\\
\underline{解答}\par
サブネットマスクが20Bitなので,サブネットマスクの別表記は以下のようになる.
\begin{center}
    {\ttfamily 255.255.240.0}\\
    {\ttfamily 11111111 . 11111111 . 11110000 . 00000000}
\end{center}
\indent つまり,ネットワークアドレス部20Bit,ホストアドレス部12Bitから構成される.\par
この場合,ネットワークアドレス部は先頭から数えて17Bit - 24Bit目に着目して考える.その中で末尾4Bitの\(0,1\)の組み合わせは全部で\(2^4\)個あり,IPアドレス先頭から17Bit - 24Bit目の64Bitを1と数えて16個先のアドレスまでネットワークアドレスであることがわかる.\par
従って,このIPアドレスのネットワークアドレスは{\ttfamily 222.229.64.0 - 222.229.79.0}である.\par
その中でもブロードキャストは,ホストアドレス部が全て{\ttfamily 1}になる必要があるので,
\begin{center}
    {\ttfamily 222.229.79.255}
\end{center}
となる.

\end{document}