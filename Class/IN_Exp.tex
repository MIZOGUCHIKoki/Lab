\RequirePackage{luatex85}
\documentclass[paper=a4,10pt,twocolumn]{jlreq}
\usepackage{amsmath,amsfonts,amssymb,mathtools,ascmac,bm,fancybox,calc,array}
\usepackage[top=20truemm,bottom=20truemm,left=15truemm,right=15truemm]{geometry}
\usepackage{graphicx,color}
\usepackage{tikz,listings,wrapfig,float,xcolor}
\usepackage{url,subcaption,multirow,framed}
\usepackage[unicode,hidelinks,pdfusetitle]{hyperref}
\usepackage{luatexja-fontspec,lltjext}
\usepackage{tcolorbox,comment}
\hypersetup{
    colorlinks=true,
    citecolor=black,
    linkcolor=black,
    urlcolor=blue
    }
    
    \usetikzlibrary{intersections,calc,arrows.meta,backgrounds,shapes.geometric,shapes.misc,positioning,fit,graphs,arrows}
    \tcbuselibrary{raster,skins,breakable,theorems}
    \setlength{\columnsep}{5mm}
    \columnseprule=0.1mm
    \ltjsetparameter{jacharrange={-2}} %日本語以外を欧文扱い

    \renewcommand{\indent}{\hspace{1\zw}}
    \setlength{\parindent}{1\zw}
    
    \renewcommand{\thefootnote}{*\arabic{footnote}}
    \renewcommand{\figurename}{Fig\ }
    \renewcommand{\tablename}{Tbl}
    \newcommand{\figref}[1]{Fig\ \ref{#1}}
    \newcommand{\tabref}[1]{Tbl\ \ref{#1}}

\makeatletter
    \renewcommand{\thefigure}{%
    \thesection.\arabic{figure}}
    \@addtoreset{figure}{section}

    \renewcommand{\thetable}{%
    \thesection.\arabic{table}}
    \@addtoreset{table}{section}

    \@addtoreset{lstlisting}{section}
\makeatother


\title{\vspace{-2cm}情報ネットワーク基礎}
\author{}
\date{}
\begin{document}
\twocolumn[
    \begin{center}
        {\LARGE \vspace{-1cm}情報ネットワーク基礎 演習}
    \end{center}
    \begin{flushright}
        {\Large K.MIZOGUCHI}
    \end{flushright}
    \dotfill\\
]
\newcommand{\ci}{◯}
\newcommand{\bt}{×}
\section*{演習1}
\begin{enumerate}
    \renewcommand{\theenumi}{\arabic{enumi}}
    \renewcommand{\labelenumi}{\textbf{{\theenumi}.}\ }
    \item 初期のイーサネット(10BASE-5)は\textbf{バス型}である.(同軸ケーブル(太):500m)
    \item OSI参照モデルに関して,各層に対応する役割は以下.
          \begin{itemize}
              \item 両端コンピュータでの処理,フロー制御:トランスポート層
              \item データの表現形式:プレゼンテーション層
              \item 途中の中継処理・経路制御:ネットワーク層
              \item データを物理信号に変換,ケーブルやコネクタの形状:物理層
              \item アプリケーション同士の会話制御:セッション層
              \item アプリケーション毎のサービス機能提供:アプリケーション層
              \item 直接接続された機器間のデータフレーム転送:データリンク層
          \end{itemize}
    \item 範囲内全メンバーに送信する通信形態を\textbf{ブロードキャスト}という.
    \item Macアドレスを確認することができる,伝統的UNIX/Linuxコマンドは,{\ttfamily ifconfig}である.
    \item OSI参照モデルに関して,1階層から順番に並べると,物理・データリンク・ネットワーク・トランスポート・セッション・プレゼンテーション・アプリケーション.
    \item 光ファイバーケーブルの特徴.
          \begin{itemize}
              \item 高速伝送が可能.
              \item \textbf{ケーブルは曲げにでない}.
              \item ケーブルが軽い.
              \item 長距離伝送が可能.
          \end{itemize}
    \item データを分割して各情報を宛先情報付きで送るのは,\textbf{パケット交換の特徴}.
    \item 光ファイバーケーブルの特徴としての正しいもの.
          \begin{description}
              \item[\bt] コア径が比較的太いマルチモード光ファイバーケーブルは,シングルモードケーブルよりも長距離通信に適している.
              \item[\bt] コア径が比較的太いマルチモード光ファイバーケーブルは,価格が安い傾向にある.
              \item[\ci] コア径が細いシングルモード光ファイバーケーブルは,長距離伝送に適している.
              \item[\bt] コア経の細いシングルモード光ファイバーケーブルは,被覆を黄色に着色してあることが多い.??
          \end{description}
    \item デュアルリンクのFDDIではある1箇所のノードで障害が発生した際,\textbf{障害箇所の手前で折り返して逆回りするために他の全ノード間で通信可能}.
    \item 通信中は回線を専有するのは,\textbf{回線交換の特徴}.
    \item かつて高速のネットワークとして利用されたFDDIは,\textbf{リング型}トポロジーで接続していた.
    \item MACアドレスでベンダーに割り当てられたOUIは上位である.
    \item MACアドレスでベンダーに割り当てられたOUIは24Bitである.{\ttfamily XX:XX:XX:YY:YY:YY,X}ベンダー
    \item 初期のイーサネット(10BASE-5)では\textbf{黄色の同軸ケーブル}を用いることが多かった.
    \item かつて高速のネットワークとして利用されたFDDIは,\textbf{サーバ}でよく利用された.
    \item イーサネットで使われているMACアドレスは,\textbf{48Bit}である.
    \item MACアドレスを確認することができるWindowsコマンドは,{\ttfamily ipconfig /all}.
    \item 通信開始時に経路を設定するのは\textbf{回線交換}の特徴.
    \item OSI参照モデルで下から3つ目はネットワーク層.
    \item 実際にデータを送っていない時も回線を専有のは,回線交換の特徴.
    \item 回線を専有しないが,混雑時には遅延が発生するのは,パケット交換の特徴.
    \item 最近のイーサネットのトポロジーはスター型.
    \item 10BASE-5のイーサネットで,稼働中のネットワークのケーブルに針を挿して接続することで,端末を増設する方式が用いられた.この方式を\textbf{ヴァンパイア}と呼ぶ.
    \item 中継地点では,その時の状況で適切な方へ送るのはパケット交換の特徴.
    \item 回線交換の特徴として,混雑時には遅延は発生しない.
    \item OSI参照モデルに関して,両端コンピュータでの処理,フロー制御の機能を有する層は,トランスポート層.
          \newpage
    \item データの到着順序が保証されないのは,パケット交換の特徴.
    \item より対線に関する正しいもの.
          \begin{description}
              \item[\bt] 長距離伝送が可能.
              \item[\ci] ケーブルが曲げに柔軟で扱いやすい.
              \item[\ci] 高速伝送が可能.
          \end{description}
    \item  OSI参照モデルに関して,直接接続された機器間のデータフレーム転送の機能を有する層は,データリンク層.
    \item パケット交換は,通信中は回線を専有しない.
\end{enumerate}
\subsection*{OSI参照モデル}
\begin{table}[h]
    \centering
    \begin{tabular}{wc{0.1\linewidth}p{0.8\linewidth}}
        \hline
        1 & アプリケーション層   \\
        2 & プレゼンテーション層 \\
        3 & セッション情報       \\
        4 & トランスポート情報   \\
        5 & ネットワーク層       \\
        6 & データリンク層       \\
        7 & 物理層               \\
        \hline
    \end{tabular}
\end{table}
\subsection*{各層の役割}
\begin{itemize}
    \item \textbf{7. アプリケーション層}\\
          アプリケーション毎のサービス提供機能
    \item \textbf{6. プレゼンテーション層}\\
          データの表現形式
    \item \textbf{5. セッション層}\\
          アプリケーション同士の会話制御
    \item \textbf{4. トランスポート層}\\
          両端コンピュータでの処理,服装抑制,信頼性提供
    \item \textbf{3. ネットワーク層}\\
          途中の中継処理,経路制御
    \item \textbf{2. データリンク層}\\
          直接接続された機器間のデータフレーム転送.
    \item \textbf{1. 物理層}\\
          データを物理的信号に変換,ケーブルやコネクタの形状.
\end{itemize}
\end{document}