\documentclass[dvipdfmx]{beamer}
\usepackage{amsmath,amsfonts,amssymb,mathtools,ascmac,bm,float,comment,url,fancybox,calc}
\usepackage[T1]{fontenc}
\usepackage[dvipdfmx]{color}
\usepackage{tikz,listings,jlisting}
\usepackage{pxjahyper}
\usepackage{xcolor}
    \usetikzlibrary{arrows}
    \usetikzlibrary{intersections,calc,arrows.meta,backgrounds,shapes.geometric,shapes.misc,positioning,fit,graphs}

            \tikzset{
                data/.style={draw,rounded corners,rectangle,text centered, minimum height = 1cm,text width=2cm},
                Key/.style={fill=red!10,rounded corners,rectangle,text centered, minimum height = 1cm,text width=2cm},
                Encryption/.style={draw,fill=blue!10,very thick,rectangle,text centered,minimum height=1cm,text width=4cm},
                empty/.style={rectangle,rounded corners, minimum height=1cm,text width=2cm}
            }

\usetheme{Madrid} 
%\usecolortheme[RGB={255,240,245}]{structure}
%\setbeamercolor{palette primary}{fg=black}
%\setbeamercolor*{palette secondary}{bg=white,fg=black}
%\setbeamercolor*{palette tertiary}{bg=white,fg=black}
%\setbeamercolor*{palette quaternary}{bg=white,fg=black}
\setbeamertemplate{navigation symbols}{}
\usefonttheme{professionalfonts}
    \renewcommand{\thefootnote}{\arabic{footnote})}
    \renewcommand{\figurename}{Fig\thesection-\thefigure}
    \renewcommand{\tablename}{Tbl\thesection-\thatable}
    \newcommand{\tabref}[1]{Tbl\thesection-\ref{#1}}
    \newcommand{\figref}[1]{Fig\thesection-\ref{#1}}
\makeatletter
    \@addtoreset{figure}{section}
    \@addtoreset{table}{section}
\makeatother
\title[暗号技術入門]{暗号技術入門}
\author[K.MIZOGUCHI]{溝口洸熙\thanks{情報セキュリティシステム研究室}}
\date{\today}
\institute[KUT]{高知工科大学 情報学群}
\titlegraphic{\includegraphics[scale=0.3]{KUTLogo.jpg}}


\begin{document}
\begin{frame}
\titlepage
\end{frame}
\begin{frame}{目次}
    \tableofcontents
\end{frame}

\section{共通鍵暗号方式}
\subsection{DES(Data Encryption Standard)}
\begin{frame}
\frametitle{目次}
\tableofcontents[currentsection,sectionstyle=show/shaded,subsectionstyle=show/shaded]
\end{frame}
\begin{frame}{DES(Data Encryption Standard)}
    \begin{block}{DES概要}
        DES暗号は,64bitの平文を64bitの暗号文に暗号化する\textbf{対称暗号}.\\
        鍵のビット長は,56bit.\footnote{64bitだが,エラー検出の情報が7Bitおきに1Bitはいるので,実質的には56Bit.}
    \end{block}
    \begin{figure}[b]
        \centering
        \caption{暗号化の概要}
        \label{fig:暗号化の概要}
        \begin{tikzpicture}
            \node[data](Plain){平文\\64Bit};
            \node[Encryption,right=0.5cm of Plain](Encryption){DESによる暗号化};
            \node[Key,below=0.5cm of Encryption](Key){鍵\\54Bit};
            \node[data,right=0.5cm of Encryption](Cipher){暗号文\\64bit};
             
            \draw[-latex,thick](Plain)--(Encryption);
            \draw[-latex,thick](Encryption)--(Cipher);
            \draw[-latex,very thick](Key)--(Encryption);
        \end{tikzpicture}
        
    \end{figure}
\end{frame}
\subsubsection{DESの構造}
\begin{frame}{DESの構造}
    \begin{block}{DESの構造}
        DESの基本構造は,ファイステルネットワーク.
    \end{block}
    \begin{block}{ファイステルネットワーク}
        \textbf{ラウンド}と呼ばれる暗号化の1ステップを何度も繰り返すようになっている.これは,多くのブロック暗号\footnote{ビット列をまとめて暗号化する暗号アルゴリズム} で採用されている.\\
    \end{block}
\end{frame}
\subsubsection{DESの脆弱性}
\begin{frame}{DESの脆弱性}
    \begin{alertblock}{DESの脆弱性}
        DESは,ブルート・フォース・アタック\footnote{可能な組み合わせを全て試す解読方法.}で現実的な時間内に解読されてしまう.\\
    \end{alertblock}
\end{frame}
\subsection{AES(Advanced Encryption Standard)}
\begin{frame}
\frametitle{目次}
\tableofcontents[currentsection,sectionstyle=show/shaded,subsectionstyle=show/shaded]
\end{frame}
\begin{frame}{AES(Advanced Encryption Standard)}
    \begin{block}{AESとは何か?}
        AES(Advanced Encryption Standard)は,これまで標準にあったDESに代わって,新しい基準となる対象暗号アルゴリズム.\\
        企業や暗号学者からAESの候補として数多の対象暗号アルゴリズムがあり,その中で\textbf{Rijndael}という対象暗号アルゴリズムが2000年にAESとして選定された.
    \end{block}
\end{frame}
\subsubsection{Rijndael}
\begin{frame}{Rijndael}
    \begin{block}{Rijndael}
        Rijndaelは,DESと同じく複数のラウンドから構成されている.DESではファイステルネットワークという基本構造が使われていたが,Rijndaelでは\textbf{SPN構造}という構造が使われている.\\
        また,Rijndaelの入力ブロックは128ビット(16バイト).
    \end{block}
\end{frame}
\subsubsection{Rijndealの暗号化}
\begin{frame}{Rijndaelの暗号化}
Rijndaelの暗号化は,4ステップに分かれている.
\begin{table}[h]
    \centering
    \begin{tabular}{llc}
        \hline
        \multicolumn{1}{c}{処理} & \multicolumn{1}{c}{意味} \\
        \hline
        SubBytes    & バイトごとの置換 \\
        ShiftRows   & 行シフト       \\
        MixColumns  & 列の混合       \\
        AddRoundKey & ラウンド鍵とのXOR \\
        \hline
    \end{tabular}
    \footnotetext[1]{このスライドで適当に定義した.}
\end{table}
\end{frame}
\begin{frame}
    \begin{figure}[h]
        \centering
        \caption{Rijndaelの暗号化}
        \label{fig:RijndaelCryptography}
        \begin{tikzpicture}
            % \tikzset{SR/.style={draw,fill=gray!10,rounded corners,text centered}};
            \node[draw,rectangle,text width=10cm,text centered](input){平文入力128ビット};
            \node[below=0.5cm of input,fill=red!10,rectangle,rounded corners,text centered,text width=10cm,minimum height=0.8cm](SB){SubBytes};
            \node[below=0.5cm of SB,fill=blue!10,rectangle,rounded corners,text centered,text width=10cm,minimum height=0.8cm](SR){ShiftRows};
            \node[below=0.5cm of SR,fill=green!10,rectangle,rounded corners,text centered,text width=10cm,minimum height=0.8cm](MC){MixColumns};
            \node[below=0.5cm of MC,fill=gray!10,rectangle,rounded corners,text centered,text width=10cm,minimum height=0.8cm](AR){AddRoundKey};
            \node[below=0.5cm of AR,draw,rectangle,text width=10cm,text centered](output){暗号文出力128ビット};
            \draw[-latex,very thick](input)--(SB);
            \draw[-latex,very thick](SB)--(SR);
            \draw[-latex,very thick](SR)--(MC);
            \draw[-latex,very thick](MC)--(AR);
            \draw[-latex,very thick](AR)--(output);
        \end{tikzpicture}
    \end{figure}
\end{frame}
\subsubsection{Rijndaelの復号}
\begin{frame}{Rijndaelの復号}
    \begin{figure}[h]
        \centering
        \begin{tikzpicture}
            \node[draw,rectangle,text width=10cm,text centered](input){平文出力128ビット};
            \node[below=0.5cm of input,fill=red!10,rectangle,rounded corners,text centered,text width=10cm,minimum height=0.8cm](SB){SubBytes};
            \node[below=0.5cm of SB,fill=blue!10,rectangle,rounded corners,text centered,text width=10cm,minimum height=0.8cm](SR){ShiftRows};
            \node[below=0.5cm of SR,fill=green!10,rectangle,rounded corners,text centered,text width=10cm,minimum height=0.8cm](MC){MixColumns};
            \node[below=0.5cm of MC,fill=gray!10,rectangle,rounded corners,text centered,text width=10cm,minimum height=0.8cm](AR){AddRoundKey};
            \node[below=0.5cm of AR,draw,rectangle,text width=10cm,text centered](output){暗号文入力128ビット};
            \draw[-latex,very thick](SB)--(input);
            \draw[-latex,very thick](SR)--(SB);
            \draw[-latex,very thick](MC)--(SR);
            \draw[-latex,very thick](AR)--(MC);
            \draw[-latex,very thick](output)--(AR);
        \end{tikzpicture}
    \end{figure}
\end{frame}
\end{document}