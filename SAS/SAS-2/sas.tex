\documentclass[aspectratio=43]{beamer}
\usepackage{luatexja-fontspec}
\usepackage{graphicx,color}
\usepackage{tikz,xcolor}
\usepackage{url}
    \usetikzlibrary{calc,arrows.meta,positioning,fit}
\usetheme{Madrid}
\usecolortheme[RGB={140,0,0}]{structure}
\setbeamertemplate{navigation symbols}{}
\setbeamercovered{transparent}
\usefonttheme{professionalfonts}
\renewcommand\familydefault{\rmdefault}
\title{SAS}
\subtitle{IoT機器向けの認証方式}
\author[K.MIZOGUCHI]{MIZOGUCHI Koki\thanks{高知工科大学 情報学群 2回生 情報セキュリティシステム研究室}}
\date{\today}
\institute[KUT]{Kochi University of Technology}
\titlegraphic{\includegraphics[scale=0.3]{../KUTLogo.jpg}}
\newcommand{\showsec}{\thesection .}
% \setbeamercolor{block title}{bg=gray,fg=white}
\setbeamercolor{block body}{bg=gray!10,fg=black}

\begin{document}
\begin{frame}
    \titlepage
\end{frame}
\begin{frame}{お品書き}
    \tableofcontents
\end{frame}
\section{自己紹介}
\begin{frame}{\showsec 自己紹介}
    \begin{block}{}
        \begin{itemize}
            \setlength{\leftskip}{2\zw}
            \item[氏名] 溝口 洸熙 (MIZOGUCHI Koki)
            \item[出身] 熊本県熊本市
            \item[年齢] 20歳(今年成人しました)
            \item[所属] 高知工科大学 情報学群 2年,Cykut\footnote{サイバーセキュリティに関する学生団体},吹奏楽部
            \item[趣味] 楽器を演奏すること.ピアノ(15年),ドラム(7年)など. \\\vspace{1em}
                {\LARGE LaTeX}が大好き.(最近Lua\LaTeX に移行した.)\\
                \dotfill
            \item[所属] アカリク:Cloud\LaTeX のオペレーションチーム.
        \end{itemize}
    \end{block}
    \begin{block}{IoTLT 歴}
        \begin{itemize}
            \item 2020年2月:IoTLT @熊本市(登壇発表)
        \end{itemize}
    \end{block}
\end{frame}
\section{ワンタイムパスワード認証方式}
\begin{frame}{\thesection .ワンタイムパスワード認証方式}
    \begin{block}{ワンタイムパスワード認証方式}
        認証毎に,認証コードが変わる認証方式.その名の通り(one-time password).身近な例では,ネット銀行の認証なんかに使われる.
    \end{block}
    \begin{columns}
        \begin{column}[t]{0.49\linewidth}
            \begin{block}<1->{利点}<1->
                \begin{minipage}[c][0.45\textheight]{\linewidth}
                    ワンタイムパスワードを盗聴され,次回認証で再利用されても,認可されない.
                \end{minipage}
            \end{block}
        \end{column}
        \begin{column}[t]{0.49\linewidth}
            \begin{block}<1->{欠点}<1->
                \begin{minipage}[c][0.45\textheight]{\linewidth}
                    \begin{itemize}
                        \item C\&R型パスワード方式
                              \begin{itemize}
                                  \item サーバからパスワードが盗取されるリスク
                                  \item クライアントのパスワードが盗視されるリスク
                              \end{itemize}
                        \item S / Key 型パスワード認証方式
                              \begin{itemize}
                                  \item 一方向性ハッシュ関数を多く使うので,処理に時間がかかる.
                              \end{itemize}
                    \end{itemize}
                \end{minipage}
            \end{block}
        \end{column}
    \end{columns}
\end{frame}
\section{SASパスワード認証方式}
\begin{frame}{\thesection .SASパスワード認証方式}
    \begin{block}{SAS(Simple And Secure password authentication protocol)}
        サーバにパスワードを知られる事なく,かつ,一方向性ハッシュ関数の利用回数の利用が少ない,軽量かつセキュアな認証方式.(清水明宏教授考案)特許取得済み.
    \end{block}
    \begin{block}{利用する演算,略記号}
        \begin{tabular}{cl}
            \(\oplus\) & 排他的論理和.\(A\oplus B \oplus B = A\)の性質がある \\
            \(E_n(x)\) & \(x\)に\(n\)回一方向性ハッシュ関数を施す               \\
            \(S\)      & パスワード(ユーザのみが知る)                         \\
            \(N_i\)    & \(i\)回目に生成された乱数
        \end{tabular}
    \end{block}
\end{frame}
\subsection{SASのバージョン}
\begin{frame}{\showsec SASのバージョン}
    \begin{block}{SASのバージョン}
        \begin{itemize}
            \item<1>    SAS
            \item<1-2>  SAS-2
            \item<1>    SAS-X(1)
            \item<1>    SAS-X(2)
            \item<1>    SAS-L
        \end{itemize}
    \end{block}
\end{frame}
\subsection{認証手順}
\begin{frame}
    \begin{center}
        {\Huge 認証手順}
    \end{center}
\end{frame}
\tikzset{PC/.style={draw,rectangle,rounded corners,text width=1.5cm,minimum height=1cm,text centered}}
\tikzset{Data/.style={rectangle,rounded corners,text width=4cm,minimum height=1cm,text centered,fill=cyan!10}}
\begin{frame}{}
    \begin{figure}[h]
        \centering
        \begin{tikzpicture}
            \node[Data,fill=magenta!10](make){\(N_i,N_{i+1}\)を生成};
            \node[Data,below=0.5cm of make](pw){秘密パスワード\(S\)入力};
            \node[Data,fill=green!10,below=0.5cm of pw,draw,dashed](s){\(E_1(S\oplus N_i)\)};
            \coordinate (server) at ($(make)+(7cm,0cm)$);
            \node[Data,draw,dashed,fill=green!10]at(server|-s)(ss){\(E_1(S\oplus N_i)\)};
            \draw[line width=6pt,draw=white](s.east)--(ss.west);
            \draw[thick,-latex,red](s.east)--(ss.west)node[midway,above]{安全なルート};
            \node[inner sep=0,fit={(make)(pw)(s)}](warp_client){};
            \node[inner sep=0,above=0.5cm of warp_client](caption_client){\ttfamily Client};
            \node[draw,rectangle,rounded corners,dotted,thick,fit={(warp_client)(caption_client)}]{};
            \node at (ss |- caption_client)(caption_server){\ttfamily Server};
            \node[draw,rectangle,rounded corners,dotted,thick,fit={(caption_server)(ss)}]{};
            \node[fill=white,rounded corners] at (ss.south east){保存};
        \end{tikzpicture}
    \end{figure}
\end{frame}
\begin{frame}{}
    \begin{center}
        {\ttfamily Client}が生成するデータ
    \end{center}
    \begin{figure}[h]
        \centering
        \begin{tikzpicture}
            \node[Data,text width=2.3cm,fill=magenta!10](m1){\(E_2(N_{i+1}\oplus S)\)};
            \node[right=0.5cm of m1](xor1){{\Huge\(\oplus\)}};
            \node[Data,text width=2.3cm,fill=magenta!10,right=.5cm of xor1](m2){\(E_1(N_i\oplus S)\)};
            \node[Data,text width=2.3cm,fill=magenta!10,below=.5cm of m1](m3){\(E_1(N_{i+1}\oplus S)\)};
            \node[right=0.5cm of m3](xor2){{\Huge\(\oplus\)}};
            \node[Data,text width=2.3cm,fill=magenta!10,right=.5cm of xor2](m4){\(E_1(N_i\oplus S)\)};
            \node[rounded corners,inner sep=0.15cm,fit={(m1)(m2)},draw,dashed](w1){};
            \node[rounded corners,inner sep=0.15cm,fit={(m3)(m4)},draw,dashed](w2){};
            \node[rounded corners,very thick,inner sep=0.5cm,fit={(w1)(w2)},draw,dotted](w3){};
            \draw[line width=7pt,-latex](w3.south)--($(w3.south)+(0,-3cm)$);
            \node at ($(w3.south)+(0,-3.2cm)$){\Large\ttfamily Server};
        \end{tikzpicture}
    \end{figure}
\end{frame}
\begin{frame}{}
    \begin{center}
        {\ttfamily Server}側の処理
    \end{center}
    \begin{figure}[h]
        \centering
        \begin{tikzpicture}
            \onslide<1->{
                \node[Data,text width=2.3cm,fill=magenta!10](m1){\(E_2(N_{i+1}\oplus S)\)};
                \node[right=0.5cm of m1](xor1){{\Huge\(\oplus\)}};
                \node[Data,text width=2.3cm,fill=magenta!10,right=.5cm of xor1](m2){\(E_1(N_i\oplus S)\)};
                \node[Data,text width=2.3cm,fill=magenta!10,below=.5cm of m1](m3){\(E_1(N_{i+1}\oplus S)\)};
                \node[right=0.5cm of m3](xor2){{\Huge\(\oplus\)}};
                \node[Data,text width=2.3cm,fill=magenta!10,right=.5cm of xor2](m4){\(E_1(N_i\oplus S)\)};
                \node[fit={(m1)(m2)(m3)(m4)},dotted,draw,thick](rd){};
                \node[below] at (rd.south){受信データ};
                \node[Data,fill=green!10,text width=2.3cm,above=0.3cm of m2](ad){\(E_1(N_i\oplus S)\)};
                \node[above] at (ad.north){認証情報};
            }
            \visible<2->{
                \node[right=0.5cm of m4](xor4){{\Huge\(\oplus\)}};
                \node[Data,text width=2.3cm,fill=green!10,right=.5 of xor4](ad2){\(E_1(N_i\oplus S)\)};
                \draw[-latex,thick](ad.east)-|(ad2.north);
            }
            \visible<3>{
                \node[rectangle,fill=gray,opacity=0.4,fit={(xor2)(m4)(ad2)},rounded corners,inner sep=0]{};
            }
        \end{tikzpicture}
    \end{figure}
\end{frame}
\begin{frame}{}
    \begin{center}
        {\ttfamily Server}側の処理
    \end{center}
    \begin{figure}[h]
        \centering
        \begin{tikzpicture}
            \node[Data,fill=green!10,text width=2.3cm,above=0.5cm of xor1](ad){\(E_1(N_i\oplus S)\)};
            \node[above] at (ad.north){認証情報};
            \onslide<1-4>{
                \node[Data,text width=2.3cm,fill=magenta!10](m1){\(E_2(N_{i+1}\oplus S)\)};
                \node[right=0.5cm of m1](xor1){{\Huge\(\oplus\)}};
            }
            \node[Data,text width=2.3cm,fill=magenta!10,right=.5cm of xor1](m2){\(E_1(N_i\oplus S)\)};
            \temporal<2-4>{
                \node[Data,text width=2.3cm,fill=magenta!10,below=1cm of m1](m3){\(E_1(N_{i+1}\oplus S)\)};
            }{
                \node[Data,text width=2.3cm,fill=magenta!10,below=1cm of m1](m3){\underline{\(E_2(N_{i+1}\oplus S)\)}};
            }{
            }
            \visible<1>{
                \node[draw,fill=white,rounded corners] at (m3.south east){\small 保存};
            }
            \visible<3-4>{
                \node at ($(m1)!0.5!(m3)$){\Huge\(\oplus\)};
            }
            \visible<4>{
                \node[inner sep=0,fit={(m1)(m3)},fill=gray,opacity=0.3,rectangle,rounded corners]{};
            }
            \only<5>{
                \draw[thick,stealth-stealth](m2.north)|-(ad.east)node[midway,right]{\Large 比較};
            }
        \end{tikzpicture}
    \end{figure}
\end{frame}
\section{共通鍵暗号方式}
\begin{frame}{\showsec 共通鍵暗号通信}
    \begin{block}{バーナム暗号は最強だ}
        \textbf{バーナム暗号}:平文\(P\)に対して鍵\(K\)を用いて暗号文\(C=P\oplus K\)を生成する.\\
        Shannonがバーナム暗号は解読不可能であることを示した.
        \begin{itemize}
            \item 仮に解読できたとしても,それが正しい平文であるかの判断が不可能であるため.
        \end{itemize}
    \end{block}
    \begin{block}{バーナム暗号の利用}
        認証情報\(E_1(N_i\oplus S)\)を鍵としたバーナム暗号で通信すれば,軽量かつセキュアな暗号通信が可能になる!
    \end{block}
\end{frame}
\section{危険性}
\begin{frame}{\showsec 危険性}
    \begin{block}{サーバ情報漏洩の危険性}
        データセンターに格納してある情報が\textbf{悪意のある管理者}または,\textbf{不正侵入}によって盗まれた場合,認可されてしまう.
        その対策として,SAS-Xがある.
    \end{block}
    \begin{block}{リプレイアタック}
        複数回認証要求すると,認証情報\(E_1(N_i\oplus S)\)が得られる脆弱性.(排他的論理和でなく,和\(+\)の演算を加えることで,改善)
    \end{block}
    \dotfill\\
    \begin{block}{相互認証}
        先日,{\ttfamily Server}・{\ttfamily Client}相互認証のプロトコルを開発.
    \end{block}
\end{frame}
\section{参考文献}
\begin{frame}{\thesection .参考文献}
    \begin{itemize}
        \item (Simple And Secure authentication protocol ver.2)[清水明宏\footnote{高知工科大学 情報学群 教授}]\\
              \url{https://www.jstage.jst.go.jp/article/itetr/26.61/0/26.61_7/_pdf/-char/ja}

    \end{itemize}
    \dotfill\\
    因みにこのプレゼンスライドは,Lua\LaTeX で作成しており,図の作成はTi{\it k}Zを用いている.
\end{frame}
\end{document}