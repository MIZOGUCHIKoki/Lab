\documentclass[t,aspectratio=169,12pt]{beamer}
\usepackage{luatexja-fontspec}
\usepackage{comment}
\usepackage{calc}
\usepackage{graphicx,color}
\usepackage{tikz,xcolor}
    \usetikzlibrary{calc,arrows.meta,shapes,positioning,fit}
    \definecolor{darkRed}{rgb}{0.5,0,0}
\usetheme{Madrid}
\usecolortheme[RGB={127,0,0}]{structure}
\setbeamertemplate{navigation symbols}{}
\renewcommand{\familydefault}{\rmdefault}
\usefonttheme{professionalfonts}
\title[SAS-M]{SAS-M{\tiny izoguchi}}
\subtitle{Simple And Secure Mutual Authentication Protocol}
\author[K.MIZOGUCHI]{MIZOGUCHI Koki\thanks{情報セキュリティシステム研究室}}
\date{\today}
\institute[KUT]{Kochi University of Technology}
\titlegraphic{\includegraphics[scale=0.3]{KUTLogo.jpg}}
\newcommand{\showsec}{\thesection\ -\ \thesubsection .}

\begin{document}
\begin{frame}
    \titlepage
\end{frame}
\begin{frame}{概要}
    \begin{block}{概要}
        SAS-M(仮)は,SAS-Lを基盤として,{\ttfamily Client}と{\ttfamily Server}が相互認証する機能を追加した認証プロトコル.
    \end{block}
    \begin{block}{相互認証の必要性}
        SAS-Lでは,{\ttfamily Server}から{\ttfamily Client}の認証は可能だが,{\ttfamily Server}から登録されている{\ttfamily Client}は{\ttfamily Server}がが正当なものとして通信が行われる.つまり,{\ttfamily Client}は{\ttfamily Server}を真には認証していない.
        \textcolor{red}{{\ttfamily Server}を真に認証するとこで,重要な情報を送る{\ttfamily Client}は{\ttfamily Server}の正当性を確かめることに十分な意味を見出せる.}
    \end{block}
\end{frame}
\subsection{認証手順}
\begin{frame}[c]
    \begin{center}
        {\Huge 認証手順}
    \end{center}
    \begin{block}{略号・記号}
        \begin{itemize}
            \setlength{\leftskip}{0.5cm}
            \item[\(\oplus\)] 排他的論理和.
            \item[\(E_n(x)\)] \(x\)に対して,\(n\)回一方向性ハッシュ関数を施す.
        \end{itemize}
    \end{block}
\end{frame}
\tikzset{Data/.style={rectangle,rounded corners,minimum height=1cm,text centered,fill=cyan!10}}
\newcommand{\aia}{\(A_i=E_1(\mathrm{SID}\mid S\oplus N_i)\)}
\newcommand{\aib}{\(A_{i+1}=E_1(\mathrm{CID}\mid S\oplus N_{i+1})\)}
\newcommand{\aic}{\(A_{i+2}=E_1(\mathrm{CID}\mid S\oplus N_{i+2})\)}
\begin{frame}{}
    \begin{center}
        {\ttfamily Server}側の処理
    \end{center}
    \begin{figure}[h]
        \centering
        \begin{tikzpicture}[remember picture]
            \tikzset{Data0/.style={Data,fill=magenta!10,text width=5cm}};
            \tikzset{Data1/.style={Data,fill=yellow!20,text width=5cm}};
            \node[Data0](a1){\aia};
            \node[Data0,below=0.5cm of a1](a2){\aib};
            \node[Data0,below=0.5cm of a2](a3){\aic};
            \node[inner sep=1mm,fit={(a1)(a2)(a3)},rounded corners,rectangle,dotted,thick,draw](warp_a){};
            \node[above=0.1cm of warp_a.north](cap_a){{\ttfamily Server}生成データ};
            \node[Data1,right=.1\textwidth of a2](m1){\(\beta_i=E_1\Big(A_i\oplus A_{i+1}\Big)\)};
            \node[Data1]at(m1|-a3)(m2){\(\beta_{i+1}=\Big(A_i\oplus A_{i+2}\Big)\)};
            \node[Data1,fill=green!10]at(m1|-a1)(ad){\LARGE\(A_i\oplus A_{i+1}\)};
            \node[inner sep=1mm,fit={(m2)(m1)},draw,dotted,rounded corners,rectangle,thick](warp_m){};
            \node[below=0.1cm of warp_m.south]{送信用データ};
            \node[inner sep=1mm,fit={(ad)},rounded corners,rectangle,draw,dotted,thick](warp_ad){};
            \node at(warp_ad.north|-cap_a){初回認証情報};
            \draw[-latex,line width=2pt](warp_ad.east)--($(warp_ad.east)+(2cm,0)$)node[midway,above]{\textcolor{red}{\ttfamily 1.Secure}}node[midway,below]{\ttfamily to Client};
            \draw[-latex,line width=2pt](warp_m.east)--($(warp_m.east)+(2cm,0)$)node[midway,above]{\ttfamily 2}node[midway,below]{\ttfamily to Client};
        \end{tikzpicture}
    \end{figure}
    \footnotetext[1]{{\ttfamily SID}:サーバ固有ID}
    \footnotetext[2]{{\ttfamily CID}:クライアント固有ID}
\end{frame}
\begin{frame}{}
    \begin{center}
        {\ttfamily Client}側の処理
    \end{center}
    \begin{figure}[h]
        \centering
        \begin{tikzpicture}[remember picture]
            \node[Data,fill=yellow!20]at(0.5\textwidth,0) (rd2){\(\beta_{i+1}=\Big(A_i\oplus A_{i+2}\Big)\)};
            \node[Data,fill=yellow!20,left=.1\textwidth of rd2] (rd1){\(\beta_i=E_1\Big(A_{i}\oplus A_{i+1}\Big)\)};
            \node[inner sep=0.3cm,dotted,draw,rectangle,rounded corners,fit={(rd1)(rd2)}](warp_r){};
            \node[fill=white]at(warp_r.north){\scriptsize 受信データ};
            \node[Data,fill=green!10,right=.1\textwidth of rd2](ad){\(\Big(A_i\oplus A_{i+1}\Big)\)};
            \node[inner sep=0.3cm,dotted,draw,rectangle,rounded corners,fit={(ad)}](warp_a){};
            \node[fill=white]at(warp_a.north){\scriptsize 認証情報};
            \node[fill=yellow!10,below=0.1\textheight of ad](ead){\(E_1\Big(A_i\oplus A_{i+1}\Big)\)};
            \draw[-latex,thick](ad)--(ead)node[midway](A){};
            \draw[thick,Stealth-Stealth](rd1)|-(ead)node[midway,fill=white,draw](B){比較};
            \shade[shading=ball, ball color=darkRed] (A) circle [radius=0.2cm]node[white]{1};
            \shade[shading=ball, ball color=darkRed] ($(B.east)!0.5!(ead.west)$) circle [radius=0.2cm]node[white]{2};
        \end{tikzpicture}
    \end{figure}
    \begin{enumerate}
        \item 認証情報に一方向ハッシュ関数を施す.
        \item \tikz[baseline=(cp1.base)]{\node[draw](cp1){比較}}で{\ttfamily Client}が{\ttfamily Server}を検証する.
              \begin{itemize}
                  \item 不一致の場合,認証不成立.({\ttfamily Server}が不正である可能性あり)
              \end{itemize}
    \end{enumerate}
\end{frame}
\begin{frame}{}
    \begin{center}
        {\ttfamily Client}側の処理
    \end{center}
    \begin{figure}[h]
        \centering
        \begin{tikzpicture}[remember picture]
            \node[Data,fill=yellow!20]at(0.5\textwidth,0) (rd2){\(\beta_{i+1}=\Big(A_i\oplus A_{i+2}\Big)\)};
            \node[Data,fill=yellow!20,left=.1\textwidth of rd2] (rd1){\(\beta_i=E_1\Big(A_{i}\oplus A_{i+1}\Big)\)};
            \node[inner sep=0.3cm,dotted,draw,rectangle,rounded corners,fit={(rd1)(rd2)}](warp_r){};
            \node[fill=white]at(warp_r.north){\scriptsize 受信データ};
            \node[Data,fill=green!10,right=.1\textwidth of rd2](ad){\(\Big(A_i\oplus A_{i+1}\Big)\)};
            \node[inner sep=0.3cm,dotted,draw,rectangle,rounded corners,fit={(ad)}](warp_a){};
            \node[fill=white]at(warp_a.north){\scriptsize 認証情報};
        \end{tikzpicture}
    \end{figure}
    \begin{equation}
        \begin{aligned}
            \beta_{i+1}\oplus \tikz[remember picture,baseline=(ad_e.base)]{\node[fill=green!10](ad_e){$\Big(A_i\oplus A_{i+1}\Big)$}} & = \Big(A_i\oplus A_{i+2}\Big)\oplus \Big(A_i\oplus A_{i+1}\Big)                                  \\
                                                                                                                                      & = \tikz[remember picture,baseline=(nad.base)]{\node[fill=blue!10](nad){$A_{i+1}\oplus A_{i+2}$}}
        \end{aligned}\tag*{(\tikz[baseline=(w3.base)]{\shade[shading=ball, ball color=darkRed] (A) circle [radius=0.2cm]node(w3)[white]{3}})}
    \end{equation}
    \begin{tikzpicture}[remember picture, overlay]
        \draw[line width=1pt,-latex](warp_a.south west)--(ad_e.north east);
    \end{tikzpicture}
    \begin{enumerate}
        \setcounter{enumi}{2}
        \item 次回認証情報,\tikz[baseline=(nad_s.base)]{\node[fill=blue!10](nad_s){\(A_{i+1}\oplus A_{i+2}\)}}を{\ttfamily Client}に保存.
        \item \(\gamma_i=E_2\Big(A_{i+1}\oplus A_{i+2}\Big)\)を生成.
        \item \(\gamma_i\)を{\ttfamily Server}へ送信.
    \end{enumerate}
\end{frame}
\begin{frame}{}
    \begin{center}
        {\ttfamily Server}側の処理
    \end{center}
    \begin{figure}[h]
        \centering
        \begin{tikzpicture}[remember picture]
            \node[Data,fill=cyan!10](rd){\(\gamma_i=E_2\Big(A_{i+1}\oplus A_{i+2}\Big)\)};
            \node[inner sep=3mm,fit={(rd)},draw,rectangle,rounded corners,dotted](warp_r){};
            \node[fill=white]at(warp_r.north){\scriptsize 受信データ};
            \node[Data,text width=1.5cm,fill=magenta!10,right=1cm of rd](sd1){\(A_i\)};
            \node[Data,text width=1.5cm,fill=magenta!10,right=1cm of sd1](sd2){\(A_{i+1}\)};
            \node[Data,text width=1.5cm,fill=magenta!10,right=1cm of sd2](sd3){\(A_{i+2}\)};
            \node[inner sep=3mm,fit={(sd1)(sd2)(sd3)},draw,dotted,rectangle,rounded corners](warp_sd){};
            \node[fill=white] at(warp_sd.north){\scriptsize 生成済みデータ};
        \end{tikzpicture}
    \end{figure}
    \begin{enumerate}
        \setcounter{enumi}{5}
        \item 生成済みデータから,\tikz[remember picture,baseline=(md.base)]{\node[fill=orange!10](md){\(E_2\Big(A_{i+1}\oplus A_{i+2}\Big)\)}}を生成する.
        \item 受信データ\(\gamma_i\)と比較する.
              \begin{itemize}
                  \item 不一致の場合,認証不成立.({\ttfamily Client}が不正である可能性あり.)
                  \item 一致した場合,認証成立.
              \end{itemize}
        \item \(\gamma_i\)を利用した共通鍵暗号方式で通信を行う.
    \end{enumerate}
\end{frame}
\begin{frame}[c]{}
    \begin{block}{軽量度}
        {\ttfamily Server}と{\ttfamily Client}の一方向ハッシュ関数の利用回数・排他的論理和の排他的論理和の演算回数は以下.\\
        \begin{table}[h]
            \renewcommand{\arraystretch}{1.4}
            \begin{tabular}{lcc}
                \multicolumn{1}{c}{演算} & {\ttfamily Client} & {\ttfamily Server} \\
                \hline
                一方向ハッシュ関数              & 3                  & 6                  \\
                排他的論理和                 & 1                  & 6                  \\
                \hline
            \end{tabular}
        \end{table}
        結果のように,{\ttfamily Client}では一方向性ハッシュ関数の適用が3回である.これは,SAS-Lの{\bfseries 0回}と比べて軽量とは言えない.{\ttfamily Client}もある程度の処理能力は必要であろう.
    \end{block}
\end{frame}
\end{document}